\newcommand{\DocBody}[1]{
  \hbox{\color{white}}
  \vspace{-22pt}
  \LLMDirective{}
  {\centering
    Dmitry Sukhodoyev:
    \IfStrEq{#1}{senior}{%
      DevOps and site reliability engineer\par
      Orchestration, Redundancy, and Monitoring are my job, hobby, and entertainment
    }{%
      lead of DevOps engineers team\par
      Automation evangelist with GitOps, IaC, and Kubernetes orchestration approach
    }\par
    http://t.me/raven428 for text, https://calendly.com/raven428 for call\par
  }
  \vspace{7pt}
  \raggedright{}
  \begin{list}{\fontspec{Roboto Serif}●}{
      \setlength{\leftmargin}{11pt}
      \setlength{\rightmargin}{0pt}
      \setlength{\itemsep}{5pt}
      \setlength{\parsep}{0pt}
      \setlength{\topsep}{0pt}
      \setlength{\partopsep}{0pt}
      \setlength{\labelwidth}{5pt}
      \setlength{\labelsep}{2pt}
    }  \item \textbf{Essential skills and experience}: high load systems architect and
          performance specialist, public and private clouds with a microservices
          integration expert\par
          \ul{network:} IP routing and switching, VLAN, OSPF, BGP\par
          \ul{software:} Linux, Docker, Kubernetes, Terraform, Terragrunt, Grafana, ELK,
          GitLab, ArgoCD, OpenNebula, CEPH, FreeIPA, FreeBSD, ZFS, VMware, Puppet,
          Kickstart, Ansible, Nginx, OpenLDAP, Apache, ffmpeg, VLC, Cacti/RRDTool, BIND,
          Postfix, Dovecot2, CometBFT-based blockchains expert\par
          \ul{databases:} MySQL, Oracle, MongoDB, Redis, PostgreSQL, Elasticsearch,
          Prometheus\par
          \ul{programming:} Perl, Shell, Python, Rust, Golang, Ruby, SQL/PL (MySQL,
          Oracle), C/C++, Git\par
          \ul{hardware:} Cisco UCS, Nexus, HP BladeSystem, Proliant, EMC Isilon, Clarion,
          Arbor\par
          \ul{clouds:} GCP, AWS, Tencent, OCI, Buildkite, GitHub Actions, GitLab CI/CD
    \item \textbf{Employment history:} {\fontspec{FreeSerif}}
          \HistoryHead{}
    \item \EmpDateRange{Aug 2021}{now}%
          \IfStrEq{#1}{senior}{%
            senior DevOps and site reliability engineer in
          }{%
            DevOps team lead of 4 engineers in
          }%
          Cosmos blockchain team, systems deployment, maintenances, lifecycle, and
          monitoring at P2P.org (Limassol, Cyprus, remote)
          \PositonHead{}
          TLDR stack: GCP, Tencent, Terraform, Terragrunt, Kubernetes, Docker, Ansible,
          Nginx, Grafana, Prometheus, GitOps, GitFlow, GitHub actions, Buildkite,
          CometBFT/Tendermint blockchains
    \item I built from scratch deployment infrastructure based on the GitOps approach
          with Ansible and Terraform modules for Terragrunt with about 90\% code testing
          coverage: “promtool test rules” for Prometheus alerts, Molecule for Ansible
          roles, test deployments to GCP for Terragrunt
    \item I collected from the whole company and migrated to infrastructure above all
          the CometBFT-based networks with their engineers. Moreover, educated them to
          deploy and improve IaC above
    \item I mentored, educated existing and new team members by voice communication
          with screen sharing and made reviews for all the pull requests to the team IaC
          GitHub repository
    \item I developed and started the on-call schedule between team members to reduce
          the load of each. By splitting duties to only updates and alerts for the
          engineer on duty when others can be free of charge to spend their time
          improving the code of infrastructure
    \item The result exceeds 60 infrastructures with a room to grow for the team with
          four engineers
          \PositonTail{}
    \item \EmpDateRange{Sep 2019}{Aug 2021}senior DevOps and site reliability engineer
          of the infrastructure team, systems, and servers deployment with maintenance at
          CloudLinux (Palo Alto, USA, remote)
          \PositonHead{}
    \item TLDR stack: AWS, Terraform, Kubernetes, Docker, Ansible, Nginx, Apache, MySQL,
          Grafana, Prometheus, Elasticsearch, Logstash, Kibana, Gitlab, GitOps, GitFlow,
          ArgoCD
    \item I achieved a dramatic speed bust for deploying new hosts by full automation
          with Ansible. I developed roles for configuring most kinds of hosts and with
          all hosts in the inventory. Of course, all the things below I did with the
          Infrastructure As Code approach in Ansible above
    \item I developed GitOps for Ansible code from GitLab. After publishing all the
          Ansible code to the default branch, it will instantly test in a VM deployed
          with Terraform in AWS and OpenNebula, then deployed to the Kubernetes cluster.
          I achieved this by GitLab pipelines with ArgoCD keeping sensitivity data in
          HashiCorp Vault
    \item I moved deprecated Zabbix monitoring by developing Prometheus alertmanager
          alerting rules with Grafana visualization. I developed integrations with
          Atlassian OpsGenie and Alerta.io both to notify different kinds of on-call
          people
    \item I collected all kinds of company backup tasks and united them to IaC. I
          developed cloud backup by uploading to available Google Drive space with
          encryption. Statuses of backup tasks collected to the monitoring system (see
          above) with notification of on-call people in case of failures
    \item I actively participated in AlmaLinux infrastructure deployment and
          architecture development
          \PositonTail{}
    \item \EmpDateRange{Sep 2017}{Aug 2019}senior DevOps and site reliability
          engineer, systems and servers deployment with maintenance at Ahrefs (San
          Francisco, USA, remote)
          \PositonHead{}
    \item TLDR stack: AWS, Terraform, Kubernetes, Docker, Puppet, Nginx, Apache,
          MySQL, Grafana, Prometheus, Elasticsearch, Logstash, Kibana, Buildkite, GitHub
    \item I developed migration with Terraform bare-metal servers to AWS infrastructure
    \item I developed Puppet master for managing 2000+ hosts of distributed systems
    \item I deployed network monitoring using ELK stack, Grafana, and Prometheus
    \item Troubleshoot integration and deployment of new projects into infrastructure
    \item Installed, upgraded, and migrated several Elasticsearch clusters with
          petabytes of data
    \item Performed mass server setup automation at Hetzner, OVH, Online, Serverius
          \PositonTail{}
    \item \EmpDateRange{Sep 2014}{Sep 2017}senior DevOps and site reliability
          engineer, systems, and servers deployment with maintenance at Kartina Digital
          GmbH (Wiesbaden, Germany, remote)
          \PositonHead{}
    \item TLDR stack: Puppet, VMware, Cisco UCS, Lua, Nginx, video processing and
          streaming, AWS, Terraform, Kubernetes, Docker, Puppet, Nginx, Apache, MySQL,
          ELK, TeamCity
    \item I achieved a centralized host configuration. Before I started working,
          all hosts were configured using pssh and pscp (parallel ssh/scp). I
          installed puppet master, collected configurations of all hosts, created roles
          for different kinds of hosts, and started configuration of all servers by
          Puppet agent
    \item I achieved a fully virtualized environment. I deployed VMware vCenter Service
          Appliance with high-availability clusters at every data center; then, I moved
          all hosts to virtual machines. Now, bare-metal servers are hypervisors. If one
          of the hypervisors physically dies, the high-availability system will restart
          all virtual machines from dead hypervisors on available hypervisors: no more
          downtimes, don’t wake up an employee. Hardware maintenance also became easier:
          someone turns the hypervisor into maintenance mode, the cluster moves virtual
          machines to other available hypervisors, and the hypervisor is ready to
          disconnect the power
    \item I achieved the full power of Cisco UCS for the company. Several of our
          hosts should be able to upload tens of gigabits per second of video streaming.
          All Cisco UCS network adapters are virtual, and the VMware ESXi hypervisor can
          move adapters between virtual machines. I merged vCenter Service Appliance with
          Cisco UCS Domains, and now numerous high-traffic hosts are virtual machines
          with pass-through and high-availability modes enabled. I described the whole
          process here: https://habrahabr.ru/post/303026 and
          here: https://yau.o6a.ru/wiki/ucs/vmware
    \item I participated in other technologies: streaming and encoding services
          by OpenResty, ffmpeg and VLC;~HP BladeSystem, Proliant, EMC Isilon, Clarion;
          converting of live video streams from any to any other types and formats:
          HTTP, UDP, TS, RTMP, FLV, RTMP, MPEG2, MPEG4, H.264, H.265; Arbor Networks
          hardware to DDoS mitigation; incremental backup, Cacti performance and
          M/Monit availability monitoring
          \PositonTail{}
    \item \EmpDateRange{Nov 2010}{Aug 2014}chief developer, systems, servers, and
          network maintenance at Digital Networks (Miass city of Chelyabinsk region,
          Russia, remote). Several hundred hosts, incremental backup, availability
          and performance monitoring, high load web and MySQL servers, virtualization
          with VMware vSphere
          \HistoryTail{}
    \item \textbf{Education:}
          \PositonHead{}
    \item Chelyabinsk State University, Bachelor of Computer Science
    \item Cisco Data Center Unified Computing Implementation (DCUCI)
    \item EMC Isilon Administration and Management (MR-1CP-ISIAM)
          \PositonTail{}
  \end{list}
  \LLMDirective{}
}
